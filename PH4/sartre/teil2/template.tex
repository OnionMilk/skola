\documentclass[12pt, letterpaper, twoside]{article}
\usepackage[T1]{fontenc}
\usepackage[utf8]{inputenc}
\usepackage[a4paper]{geometry}
\usepackage{array}
\usepackage{booktabs} % For prettier tables
\usepackage{multirow}
\usepackage{multicol}
\usepackage{ragged2e}
\usepackage{xcolor}
\usepackage{gensymb}
\usepackage{fullpage}
\usepackage{hyperref}
\usepackage{amsmath}
\usepackage{scrextend}
\usepackage{graphicx}
\usepackage{enumitem}
\usepackage{blindtext}
\usepackage[ngerman]{babel}

% \graphicspath{ {./bilder/} }
% \newcommand{\cfootnote}[1]{\footnote{\centering #1}}
\title{Das Menschenbild der atheistischen Existentialismus Teil II}
% \author{Simon Freiermuth \\ \href{mailto:simon@freiermuth.org}{simon@freiermuth.org}}
\date{\today}

\begin{document}

%\begin{titlepage}
\maketitle
%\end{titlepage}

\begin{flushleft}

% \chapter*{Das Menschenbild der atheistischen Existentialismus Teil II}

\section{Aufgabe: Selbstentwurf}

Sartres Selbstentwurf basiert auf einer einheitlichen Struktur des menschlichen Handelns:
Vergangenheit: Diese schafft die Motive für das jetzige Handeln eines Menschen, bspw. durch Erlebnisse.
Gegenwart: In der Gegenwart findet das konkrete Handeln statt; dieses ist auf die Zukunft ausgerichtet.
Zukunft: Der Mensch antizipiert die Zukunft in seiner Vorstellungskraft und löst sich damit von
Gegenwart und Vergangenheit. Diese Fähigkeit nennt Sartre Transzendenz; sie ermöglicht erst freies Handeln.


\section{Aufgabe: Determinismus (alles läuft nach dem Kausalgesetz)}

Biologismus: Bezeichnet die Übertragung biologischer Masstäbe, Begriffe und Gesichtspunkte auf andere
Wissensgebiete. Hier entsprechend: die biologischen Voraussetzungen bestimmen die menschliche Natur.
Dazu zählen bspw. die Versuche im 19. und anfangs 20. Jahrhundert, Kriminelle anhand biologischer
Merkmale zu identifizieren.
Soziologismus: Dieser Begriff bezeichnet die Auffassung, der Mensch sei in seinem Fühlen und Handeln,
seinen Fähigkeiten und Bedürfnissen nur durch die Gesellschaft geprägt.
In Abgrenzung zum Biologismus ergibt sich hier die Kriminalität rein aus den sozialen Umständen,
dem Zustand der Gesellschaft.

\pagebreak

\section{Aufgabe: Einwände der Deterministen}

\begin{enumerate}[label=\textbf{\alph*)}]
	\item
	Argumente Sartres: Gemäss den Deterministen scheint der Mensch „gemacht zu werden“,
	als Konsequenz verschiedener Umstände, bspw. biologischen oder gesellschaftlichen Voraussetzungen.
	Sartre stellt dem entgegen, dass es nicht die Widrigkeiten oder Umstände als solche sind,
	welche die Freiheit einschränken, sondern es ist das freie Setzen eines Zwecks,
	welche aus einem Umstand eine Schwierigkeit macht.
	Während normalerweise derjenige als frei bezeichnet wird, der seine Entwürfe und Ziele erreichen
	kann, stellt Sartre fest, dass eine solche Welt ähnlich der eines Traums wäre, wo sich das Mögliche
	nicht mehr vom Realen unterscheidet. Er argumentiert, dass ein freier Mensch nur
	engagiert in einer Widerstand leistenden Welt möglich ist. Sein Freiheitsbegriff bezieht sich auf die
	Wahl, nicht das Erreichen des erwünschten Resultats.


	\item
	Inwiefern sind folgende Menschen in Sartres Verständnis frei:\\
	Obdachloser: Ist frei, sich eine Wohnung zu wünschen und entsprechend engagierte Schritte zu
	unternehmen. Kann aber auch frei sein, kriminell zu werden,
	eine Arbeit suchen oder als Obdachloser leben zu wollen.\\
	Tetraplegiker: Er ist frei, sich zu bewegen zu versuchen oder sich einen anderen Zweck zu wünschen.
	Gefangener kurz vor Hinrichtung: Er ist frei zu versuchen, sich zu wehren, auszubrechen,
	oder sich selbst umzubringen.
	Mensch mit Zwangsneurose (z.B. Waschzwang): Er ist frei, Zwecke zu setzen, bspw. Sauberkeit der Kleider,
	kann auch versuchen sich dagen zu wehren und versuchen den Zwang zu ingnorieren.


	\item
	Auseinandersetzung
	Aus meiner Sicht ist Sartres Idee der Freiheit zu absolut. Sie setzt ein unbegrenztes Bewusstsein
	voraus,	das es erlaubt, Zwecke zu setzen und immer zu wählen.
	Dieses Bewusstsein ist unrealistisch,
	bspw. durch unterschiedliche intellektuelle Kapazitäten oder durch Prozesse, die im Unterbewusstsein
	ablaufen (siehe Freud). Gleichzeitig finde ich den Ansatz gut mit Bezug auf die Verantwortung des Einzelnen.
	Durch die absolute Freiheit entfallen alle Entschuldigungen für das eigene Verhalten.
	Im Determinismus kann man sich zu sehr hinter einer Art „Bestimmung“ oder den Umständen verstecken.
	Entsprechend überzeugt Sartres Idee von Freiheit als Basis oder Grundlage, muss aber mit tatsächlichen
	Sachzwängen, dem Einfluss des Unterbewusstseins u.ä. relativiert werden. Es ist leicht vorstellbar,
	dass viele Menschen mit der absoluten Freiheit nach Sartre überfordert sind.

\end{enumerate}

\end{flushleft}

\end{document}
