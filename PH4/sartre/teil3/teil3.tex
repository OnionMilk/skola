\documentclass[12pt, letterpaper, twoside]{article}
\usepackage[utf8]{inputenc}
\usepackage[a4paper]{geometry}
\usepackage{array}
\usepackage{booktabs} % For prettier tables
\usepackage{multirow}
\usepackage{multicol}
\usepackage{ragged2e}
\usepackage{xcolor}
\usepackage{gensymb}
\usepackage{fullpage}
\usepackage{hyperref}
\usepackage{amsmath}
\usepackage{scrextend}
\usepackage{graphicx}
\usepackage{enumitem}
\usepackage[ngerman]{babel}

\graphicspath{ {./bilder/} }
\newcommand{\cfootnote}[1]{\footnote{\centering #1}}
\title{Teil III: Konsequenzen für die Ethik und die Bedeutung der Urwahl}
\author{Simon Freiermuth \\ \href{mailto:simon@freiermuth.org}{simon@freiermuth.org}}
\date{\today}

\begin{document}

%\begin{titlepage}
\maketitle
%\end{titlepage}

\begin{flushleft}




\textbf{Antwort von I. Kant bzw. J-P. Sartre auf den Pastor}

\hfill

Beide würden widersprechen. Immanuel Kant würde seinen kategorischen Imperativ einbringen,
dass man nur nach derjenigen Maxime handeln soll, durch die man zugleich wollen kann,
dass sie ein allgemeines Gesetz wird. Dafür braucht es keinen Glauben, aber es entsteht trotzdem eine Moral,
und die Verantwortung für das eigene Handeln wird klar. Bei Sartre findet der Mensch keinen Halt
bei anderen Instanzen, alles ist erlaubt. Das hat der Pastor vielleicht im Kopf gehabt bei seiner Aussage.
Bei Sartre ergibt sich aber aus der radikalen Freiheit („alles ist erlaubt“) auch eine absolute Verantwortung.
Der Mensch muss sich vor sich selbst für sein Tun verantworten, es gibt keine Entschuldigungen.
Er muss auch selbst die Richtung wählen, ohne Hilfe von einem Zeichen oder von Gott.\\

\hfill

\textbf{Dilemma eines Schülers}

\begin{enumerate}[label=\alph*)]

\item
In der Ausgangslage gibt es für den Schüler keine neutrale
Position, er muss zwischen zwei Handlungsoptionen, zwei Wertesystemen
(oder Typen von Moral) entscheiden. Die beiden möglichen Handlungen sind
so unterschiedlich, dass sie kaum vergleichbar oder bewertbar sind.
Für Sartre ist er verurteilt zur Freiheit, d.h. zur eigenständigen Wahl.
Wenn er sich an einer aussenstehenden Moral bspw. der christlichen
Nächstenliebe ausrichten würde, hätte er trotzdem noch das Problem
der Bewertung - Bruder oder Mutter. Wenn er sich von einem inneren Gefühl leiten liesse, würde sich dieses Gefühl aus
den Handlungen aufbauen, und dann wären auch das Entscheide, die ihm voll zuzurechnen wären.

\item
Beispiele: Wahl eines Lebenspartners, der von den eigenen Eltern abgelehnt wird; Wahl eines Berufsweges, der aus Sicht der Eltern unpassend oder nicht erfolgversprechend ist.

\item
Beim Utilitarismus ist eine Handlung moralisch richtig, wenn sie den Gesamtnutzen maximiert,
das grösstmögliche Glück erzielt. Wie sollte dieser Gesamtnutzen ermittelt werden,
wie sollten verschiedene Alternativen bewertet werden können (vgl. das Dilemma des Schülers)? Was ist Glück für den Einzelnen?
In der Pflichtethik von Kant können bestimmte Handlungen als absolut gut oder schlecht bezeichnet werden. Entscheidend ist, ob sie einer verpflichtenden Regel gemäss begangen werden.
In der Sichtweise von Sartre sind auch diese Regeln freie Entscheide von Menschen und die Beantwortung der
obgenannten Fragen kann nur der Einzelne für sich vornehmen. Beide Philosophien sind unfähig,
eine für alle Situationen geltende, konkrete Handlungsanweisung zu geben, weshalb letztlich die radikale Freiheit nach Sartre zur Geltung kommt. Der Mensch muss aus sich selbst heraus entscheiden.

\end{enumerate}

\hfill

\textbf{Bedeutung der „Urwahl“}

\hfill

\begin{enumerate}[label=\alph*)]

\item
Die Persönlichkeit der Frau wurde durch die beschriebenen Prozesse zwar nicht determiniert (im Sinne von bestimmt),
aber sicher stark beeinflusst. Auch unter den erlebten Umständen sind verschiedene Entwicklungen und entsprechend
verschiedene Persönlichkeiten denkbar. Die Frau nimmt selbst durch ihre Entscheidungen (auch) Einfluss.
Sie kann sich die Mechanismen auch bewusst machen und damit aktiv umgehen.

\item
Sie für ihr Leben verantwortlich, zumindest ab dem Erwachsenenalter, bzw. der Möglichkeit,
unabhängig zu leben. Die Frau entscheidet selbst darüber, ihre Art zu behalten oder zu verändern.
Da sie offenbar den Psychotherapeuten aufgesucht hat, will sie dies auch aktiv tun.

\item
Aus meiner Sicht ist diese „Urwahl“ ein Versuch, die praktische Unzulänglichkeit
von Sartres Theorie der absoluten Freiheit zu korrigieren. Eine solche Urwahl setzt voraus,
dass das Gehirn eines Menschen bereits bei Geburt voll ausgebildet ist. Dies scheint mir nicht
der Fall zu sein. Auch wenn nicht alles Unbewusste von vorneherein unbewusst ist oder für immer
unbewusst bleibt, so gehe ich doch von einer Entwicklung des Gehirns und des Menschen im Laufe des Lebens aus.
Entsprechend finde ich dieses Konzept der Urwahl nicht sehr überzeugend.

\item
Ich denke, Sartre würde ihr zu 100\% sämtliche Verantwortung zumessen. Die Frau hat durch ihre Urwahl und
durch die völlig freien Entscheide für Handlungen im Laufe der Zeit sich selbst geschaffen.

\end{enumerate}
\end{flushleft}

\end{document}
