\documentclass[12pt, letterpaper, twoside]{article}
\usepackage[utf8]{inputenc}
\usepackage[a4paper]{geometry}
\usepackage{array}
% \usepackage{booktabs} % For prettier tables
% \usepackage{multirow}
% \usepackage{multicol}
\usepackage{ragged2e}
% \usepackage{xcolor}
\usepackage{gensymb}
\usepackage{fullpage}
\usepackage{hyperref}
% \usepackage{amsmath}
\usepackage{scrextend}
% \usepackage{graphicx}

% \graphicspath{ {./bilder/} }
%\newcommand{\cfootnote}[1]{\footnote{\centering #1}}

\title{En jämförelse av Hjalmar Söderbergs \textit{Gertrud} och Gustav Frödings \textit{Samlade Dikter}}
\author{Simon Freiermuth \\ \href{mailto:simon@freiermuth.org}{simon@freiermuth.org}}
\date{\today}

\begin{document}

%\begin{titlepage}
\maketitle
%\end{titlepage}


\textit{Gertrud}, Skriven av Hjalmar Söderberg, utgavs 1906:
Ett drama som utspelar sig över 3 akter.\\
Det handlar mest om hur relationer går sönder.\\
Gertrud är en kvinna ur övre medelklassen i Stockholm, hon är gift med Gustav Kanning,
som ska bli statsråd. Hon har tidigare varit gift med Gabriel Lidman,
en författare.\\
Hennes kärlek till Lidman verkade fullkomlig, men när Lidman gifte sig med Gertrud ville
han bli berömd och rik, hans hjärta drog honom till sina verk.
Gertrud länmar honom trots sin kärlek, för hans skull,
så att han ostört kan fullfölja sin bestämmelse som konstnär.
Hon lämnar sedan även Kanning för att ha någon slags relation med Erland Jansson,
en ung kompositör, men det visar sig att det inte funkar heller,
eftersom Jansson inte älskar henne på riktigt.\\

Språket är gammaldags, för det mesta formellt, men i privata sammanhang svärs det ibland.
(\textit{Kanning: ``Se så för fan''})

\hfill

Dikterna i \textit{Samlade dikter} utgavs 1891-1914.
Dikterna är melodiska, vissa korta, andra långa.
Jag har valt ut \textit{En Morgondröm}, en av det mest kända dikterna.\\
När dikten gavs ut 1896 blev det en uppståndelse, den ansågs vara respektlös mot kvinnor,
Fröding blev åttalad och frikänd, men hans redan dåliga psykiska hälsa rubbades ännu mer.

I \textit{En Morgondröm} skildras En erotik som Fröding själv kanske aldrig upplevde.\\
\textit{En Morgondröm} handlar om en dröm om en dunderpotent ung man som rör sig igenom naturen.
Han har sex med en flicka som hittar honom i skogen. Det hela är absurt och konstigt för dagens
standard, det var extremt oanständigt 124 år sen när dikten gavs ut.
På den tiden sågs den som nervärderande mot kvinnor, någonting som Fröding var känslig för, men jag
tror inte han menade det så, han försökte nog bara beskriva ultimativ (sexuell) frihet.\\

Miljön är också absurd, folk springer omkring nakna i naturen. Det var kanske så Rousseau tänkte sig
Människans naturtillstånd ser ut. Mannen som beskrivs passar överens med Rousseaus ``ädla vilda''.\\
Fröding var bildad i historia, jag tror att der är därifrån han fick sin inspiration.

Dikten är i obunden form, såklart, innehållet är ju inte särkilt bundet heller.

Alla karaktärer är fria i dikten, det finns inget samhälle som kontrollerar någon.

I dikten sägs det inte mycket, alla karaktärerna bara strövar omkring på måfå och följer sina instinkter.
Diktarjaget är berättarrösten och har ett bildligt och svepande språk, det beskrivs bildligt hur
det går till när köttets lust ska utlevas.

\begin{flushleft}

En central fråga i \textit{Gertrud} är ``vad är kärlek?''\\

Ingen av det två(/tre) männen som älskar Gertrud verkar förstå henne. Kanning vill ha en vanlig
äktenskap, där han kan vara politiker, där kärleken är kontrollerad och ofarlig,
där storhet och ära är viktiga.\\
Äktenskapet är ett sett att slippa vara ensam.
Man kan se på konversationen som Lidman och Kanning för mot slutet av dramat, Kanning babblar på om sina
politiska bekymmer medan Lidman oentusiastiskt säger \textit{Verkligen?}. Kannings värld är enklare,
han pratar om hur filosofer säger det de tänker och politiker säger det som är best i situationen,
medan Lidman har mycket mer nyanserade föreställningar om hur saker håller till.
Jansson är enbart fokusserad på köttets lust, han verkar ganska ointressant, han vill inte ens ha
någon seriös relation.

\hfill

Gertrud kan bara acceptera fullkomlig kärlek, hon går inte med på Lidmans idé om själens obotliga ensamhet,
hon vill inte vara ensam, därför lämnar hon alla.

Gertrud verkar ha älskat Lidman och Jansson, men inte Kanning. Hon gifte sig Med Kanning i sökan efter
efter köttets lust. Sonen hon fick skulle ha gett relationen mening, men eftersom han dog finns det ingen
anledning för Gertrud att stanna.

\hfill

Kanning har inga djupsinniga repliker som Gertrud eller Lidman har, allt han säger handlar om politiken,
hur mycket han behöver Gertrud, och om vad han ska göra härnäst, aldrig om kärlekens betydelse, eller
om skillnaden mellan ren lust och själsligt sammanhållande.
Lidmans repliker handlar om hur han är trött på berömmen han får och hur den status han har uppnått inte
betyder någonting, eftersom han är ensam, hans kärlek till Gertrud är det enda viktiga i hans liv.


Jansson är en typisk Flanör, han strävar omkring i Stockholm, han går till fester, han ``lefver rövare''
på nätterna, han vet inte riktigt vad han ska göra med sig själv. I många av Söderbergs texter,
som till exempel \textit{Martin Bircks ungdom}, är Flanören den centrala figuren, men inte här.

Gabriel Lidman flanerar också, han kan vet inte heller vad han ska med sig själv.

I \textit{Gertrud} är inte Flanerandet framhävd så mycket.

Dramat utspelar sig i Stockholm, i en till skillnad ifrån \textit{En Morgondröm} nästan överciviliserad miljö,
med överklassen och övre medelklassen som dominerar, samma kretsar som Söderberg själv rörde sig i.

\hfill

Efter sekelskiftet, när \textit{Gertrud} skrevs var det modernt med flanörstämmningen\\
Söderberg själv ansågs vara flanör, i början av sin karriär.

Lidman är personen som man kanske
tänker på först, Lidman är en författare, precis som Söderberg, han har gift om sig
och han har flyttat ifrån Sverige.
Söderberg flyttade visserligen bara till Danmark och inte ända till Italien.
Kanske ville Söderberg illustrera ett dilemma han själv hade, när han lät Lidman skriva orden:\\
\textit{Kvinnans kärlek och mannens verk - de två äro fiender af begynnelsen.}\\


Jag tror karaktären som är Söderberg mest lik är Gertrud, Söderberg har precis som Gertrud gift
sig med en person, sedan skiljt sig efter att ha blivit störtförälskad i någon
annan\footnote{\url{https://litteraturbanken.se/f\%C3\%B6rfattare/S\%C3\%B6derbergH/presentation}}.
Till skillnad ifrån Gertrud levde Söderberg med sin andra kärlek tills han dog.\\

Söderberg skrev dramat som katharsis, han hade själv kärleksproblem, man kan jämföra
Söderbergs egna liv med olika karaktärers liv i boken. När Söderberg skrev dramat över
sommaren 1906 i Köpenhamn hade hans romans med Maria von Platen slutat.\\


Söderberg var en kronisk pessimist, någonting man ser i Gertruds öde.
Gertrud ville hitta inget mindre än ``sann kärlek'', någonting som hon inte riktigt hade med Kanning,
och definitivt inte med Jansson.
Däremot såg hennes relation till Lidman mer ut som äkta kärlek, ända tills Lidmans trosbekännelse:\\
\textit{Jag tror på köttets lust och själens obotliga ensamhet.}\\

\hfill

Gertrud\\
\textit{Du sa orden i en stund, då jag trodde att vår lyckodröm var full och hel verklighet.
Och du väckte mig ur den med det orden.}

\textit{när det hade gått sönder för oss två, sökte jag köttets lust hos en man som var främmande för min själ.}


\hfill
%Fröding

Söderbergs zitat ``Jag tror på köttets lust och själens obotliga ensamhet'' passar med Frödings liv,
med tanke på Frödings två år efter universitetet då han ``levde rövare'' i Uppsala och slösade bort
sina 17'000 kr som han fick i arv\footnote{\url{https://sok.riksarkivet.se/sbl/mobil/Artikel/14556}}.

Sann kärlek såsom Gertrud vill ha den är omöjlig att nå, hon verkar tycka att man inte kan hålla
på med någonting annat en varann om man älskar varann.

% Avslutande mening om Gertrud
Gertrud är dömd till det eviga letandet,\\
efter fullkomlig kärlek.

\end{flushleft}

\end{document}
