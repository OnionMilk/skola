\documentclass[12pt, letterpaper, twoside]{article}
\usepackage[utf8]{inputenc}
\usepackage[a4paper]{geometry}
\usepackage{array}
\usepackage{booktabs} % For prettier tables
\usepackage{multirow}
\usepackage{multicol}
\usepackage{ragged2e}
\usepackage{xcolor}
\usepackage{gensymb}
\usepackage{fullpage}
\usepackage{hyperref}
\usepackage{amsmath}
\usepackage{scrextend}
\usepackage{graphicx}

\graphicspath{ {./bilder/} }
%\newcommand{\cfootnote}[1]{\footnote{\centering #1}}

\title{En jämförelse av Hjalmar Söderbergs \textit{Gertrud} och Gustav Frödings \textit{Samlade Dikter}}
\author{Simon Freiermuth \\ \href{mailto:simon@freiermuth.org}{simon@freiermuth.org}}
\date{\today}

\begin{document}

%\begin{titlepage}
\maketitle
%\end{titlepage}

\textit{Gertrud} utgavs 1906

Dikterna i \textit{Samlade dikter} utgavs 1891-1914

\abstract{
\textit{Gertrud}:
Ett drama som utspelar sig över 3 akter.
Gertrud är en kvinna ur övre medelklassen i Stockholm, hon är gift med Gustav Kanning, som ska bli statsråd. Hon har tidigare varit gift med Gabriel Lidman,
en författare.

\textit{Samlade dikter}:
}

\begin{flushleft}


\hfill

% Freud
En central fråga är ``vad är kärlek?''\\

\textbf{Jag tror på köttets lust och själens obotliga ensamhet.}

Katharsis för Söderberg

Gertrud verkar ha älskat Lidman och Jansson, men inte Kanning.

	\# Söderberg $\rightarrow$ Kanning, Söderberg $\rightarrow$ Lidman\\
Man kan jämföra Söderbergs egna liv med olika karaktärers liv i boken. Lidman är personen som man kanske tänker på först,
Lidman är en författare, precis som Söderberg, han har gift om sig och han har flyttat ifrån Sverige.
Söderberg flyttade visserligen bara till Danmark och inte ända till Italien\\

	\# Jansson: Flanör\\
efter sekelskiftet, \dots flanörstämmning\\

Man och kvinna är en av Frödings längre dikter. den visar frödings kvinnosyn och hans religiösa natur.

En Morgondröm % En erotik som fröding själv kanske aldrig upplevde

\end{flushleft}

\end{document}
