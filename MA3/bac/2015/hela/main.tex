\documentclass[12pt, letterpaper, twoside]{article}
\usepackage[utf8]{inputenc}
\usepackage[a4paper]{geometry}
\usepackage{array}
\usepackage{booktabs} % For prettier tables
\usepackage{multirow}
\usepackage{multicol}
\usepackage{ragged2e}
\usepackage{xcolor}
\usepackage{gensymb}
\usepackage{fullpage}
\usepackage{hyperref}
\usepackage{amsmath}
\usepackage{scrextend}
\usepackage{graphicx}
\usepackage{enumitem}
\usepackage{multicol}

\graphicspath{ {./} }
\newcommand{\cfootnote}[1]{\footnote{\centering #1}}
\title{Bac 2015 Del A}
\author{Simon Freiermuth \\ \href{mailto:simon@freiermuth.org}{simon@freiermuth.org}}
\date{\today}

\begin{document}

%\begin{titlepage}
\maketitle
%\end{titlepage}

\begin{flushleft}

\begin{enumerate}[label=\textbf{\arabic*)}]
%------------------------------------------------------------------------------------------
%   1)
%------------------------------------------------------------------------------------------
    \item
    Lös eqvationen:\\
    $ln(3x-14)=0$\\

    \hfill

    \textcolor{red}{
    $ e^{ln(3x-14)}\ =\ e^0 $\\
    $ 3x-14=1 $\\
    $ x=\dfrac{15}{3}=5 $
    }

    \hfill

%------------------------------------------------------------------------------------------
%   2)
%------------------------------------------------------------------------------------------
   \item

   \hfill

   % \includegraphics[scale=0.08]{graf1_tangent.png}\\
   Bestäm eqvationen för tangtenten till grafen $ f $ där $ x=-1 $.\\

   \textcolor{red}{
      $ f'(-1)=-4 $\\
      $ 3=-4*(-1)+m $\\
      $ m=3-4 = 1 $\\
      $ t(x) = -4x+1 $
   }

   \hfill

%------------------------------------------------------------------------------------------
%   3)
%------------------------------------------------------------------------------------------
   \item
   funktionen $ f $ är given av $ f(x)=\dfrac{1}{3}x^3-\dfrac{1}{2}x^2-2x+\dfrac{1}{3} $.

   Bestäm det intervall i vilket $ f $ är avtagande.

   \textcolor{red}{
      $$ f'(x) = x^2 - x - 2 $$\\
      $$ f(x) = 0 $$\\
      $$ 0 = x^2 - x - 2 $$\\
      $$ (x-\frac{1}{2})^2 - \frac{1}{4}-2 = y $$\\
      $$ (x-\frac{1}{2})^2 + \frac{-9}{4} = 0 $$\\
      $$ (x-\frac{1}{2})^2 = \frac{9}{4} $$\\
      $$ \sqrt{(x-\frac{1}{2})} = \frac{9}{4} $$\\
      $$ x-\frac{1}{2} = \sqrt{\frac{9}{4}} $$\\
      $$ x-\frac{1}{2} = \pm\frac{3}{2} $$\\
      $$ x = 2\ eller\ x=-1 $$
      \hfill\\
    \begin{center}
        \begin{tabular}{ |c|c|c|c|c|c| }
           \hline
                     & $ x<-1 $ & $ x=-1 $ & $ -1<x<2 $ & $ x=2 $ & $ 2 < x $  \\
           \hline
           $ f'(x) $ &    +     & 0        & -          & 0       &  +         \\
           \hline
           $ f(x) $  & $\nearrow$ & -      & $\searrow$ & -       & $\nearrow$ \\
           \hline
        \end{tabular}
    \end{center}
    $ f $ är avtagande mellan $ x>-1 $ och $ x<2 $.
   }

   \hfill

%------------------------------------------------------------------------------------------
%   4)
%------------------------------------------------------------------------------------------
   \item
   Funktionen $ f $ är given av $ f(x)=1-\dfrac{3}{x+2}\ ,\ x>-2 $.\\
   \hfill\\
   Bestäm den primitiva funktionen $ F $ till $ f $ som uppfyller $ F(3)=0 $.

   \textcolor{red}{
      $$  \int(1-\frac{3}{x+2})dx $$
      $$ F(x) = x-3*ln(x+2) + c $$
      $$ 0 = 3-3*ln(3+2) + c $$
      $$ 0 = 0 + c $$
      $$ F(x) = x-3*ln(x+2) $$
      $$ F(x) =  x- 3*ln(x+2) + c $$
      $$ F(3) =0  $$
    $$ F(x) =x- 3*ln(x+2) + c $$
    $$ 0 = 3-3*ln(3+2) + c $$
   }


   \begin{multicols}{2}
	[
       \item
	Diagrammet visar grafen till en funktion $ g $.\\
	Följande integraler är givna:\\
	]

	$$ \int^5_0 g(x)dx = 1.6 $$
	$$ \int^5_2 g(x)dx = 3.6 $$
	$$ \int^8_2 g(x)dx = -3.4 $$

	\columnbreak

	% Jag ska sätta in en annan bild när jag har löst uppgiften
	\includegraphics[scale=0.05]{graf1_tangent.png}

    \end{multicols}


\end{enumerate}

\end{flushleft}

\end{document}
