\documentclass[12pt, letterpaper, twoside]{article}
\usepackage[utf8]{inputenc}
\usepackage[a4paper]{geometry}
\usepackage{array}
\usepackage{booktabs} % For prettier tables
\usepackage{multirow}
\usepackage{multicol}
\usepackage{ragged2e}
\usepackage{xcolor}
\usepackage{gensymb}
\usepackage{fullpage}
\usepackage{hyperref}
\usepackage{amsmath}
\usepackage{scrextend}
\usepackage{graphicx}
\usepackage{tikz}
\usepackage{enumitem}
\usepackage{chemfig}
\usepackage{chemformula}

\graphicspath{ {./bilder/} }
\title{Vecka 19, Lektion 3\\ Bac 2015 B1}
\author{Simon Freiermuth \\ \href{mailto:simon@freiermuth.org}{simon@freiermuth.org}}
\date{9 Maj, 2020}

\begin{document}

%\begin{titlepage}
\maketitle
%\end{titlepage}

\begin{flushleft}



\hfill

\begin{enumerate}[label=\textbf{\alph*)}]
%%%%%%%%%%%%%%%%%%%%%%%%%%%%%%%%%%%%%%%%%%%%%%%%%%%%%%%%%%%%%%%%%%%%%%%%%%%%%%%%%%%%%%%%%%%%%%
%   a)
%%%%%%%%%%%%%%%%%%%%%%%%%%%%%%%%%%%%%%%%%%%%%%%%%%%%%%%%%%%%%%%%%%%%%%%%%%%%%%%%%%%%%%%%%%%%%%
    \item % a)
    $ A $, $ B $, $ C $ och $ D $ är organiska ämnen.\\
    $ B $, $ C $, $ D $ kan utvinnas ifrån $ A $.

    \hfill

    \begin{enumerate}[label=\textbf{\roman*. }]
        \item % i.
        Ange systematiska (IUPAC) namnet av ämnet $ A $.

        \textcolor{red}{
            Butanol:\\
           \hfill \\
            \chemfig{
                C(-[:0]
                    C(-[:0]
                        C(-[:0]
                            C(-[:0]OH)(-[:90]H)(-[:270]H)
                        )   (-[:90]H)(-[:270]H)
                    )   (-[:90]H)(-[:270]H)
                )   (-[:90]H)(-[:180]H)(-[:270]H)
            }
        }

        \hfill % ii.

        \item
        Ange strukturformeln och det systematiska namnet (IUPAC) av
        ämnet $ B $.

        \textcolor{red}{
            Buten: \\
            \hfill \\
            \chemfig{
                C(-[:0]
                    C(-[:0]
                        C(=[:0]
                            C(-[:315]H)(-[:45]H)
                        )   (-[:270]H)
                    )   (-[:90]H)(-[:270]H)
                )   (-[:90]H)(-[:180]H)(-[:270]H)
            }
        }

        \hfill

        \item % iii.
        Namnge reaktionen som sker när $ A $ omvandlas till $ B $.

        \textcolor{red}{
            Dehydreringsreaktion. I en dehydreringsreaktion så avges en vatten molekyl och det bildas en alken. Alkenen innehåller en dubbelbindning.
        }

        \pagebreak

        \noindent Ämnet $ C $ ger ett positiv resultat med fehlings test.

        \hfill

        \item % iv.
        Ge strukturformeln och det systematiska namnet (IUPAC) för ämnet $ C $.

        \hfill

        \textcolor{red}{
            Butanal: \\
            \hfill \\
            \chemfig{
                C(-[:0]
                    C(-[:0]
                        C(-[:0]
                            C(-[:315]H)(=[:45]O)
                        )   (-[:90]H)(-[:270]H)
                    )   (-[:90]H)(-[:270]H)
                )(-[:90]H)(-[:180]H)(-[:270]H)
            }
        }

%\chemname[<dim>]{\chemfig{<code of the molecule>}}{<name>}

% C (bindning[:vinkel]atom) (bindning[:vinkel]atom) (bindning[:vinkel]atom) (bindning[:vinkel]atom)

        \hfill

        \item % v.
        Ge halveqvationen för reduktionen av en av det oxiderande agenterna som
        ofta används i dessa reaktionerna.


        \textcolor{red}{
            \textit{$ \ch{KMnO4} \rightarrow \ch{K+} + \ch{MnO4-} $}\\
            \textit{$ \ch{8 H+} + \ch{7 e-} + \ch{MnO4-} \rightarrow \ch{4 H2O} + \ch{Mn^2+} $}
        }

        \item % vi.
        Sortera ämnena $ A $, $ B $ och $ C $ i stigande ordning enligt kokpunkten

        \begin{enumerate}[label=\textbf{\arabic*. }]
        \color{red}
                \item
                $ B $ (Buten)Buten har lägst kokpunkt eftersom den saknar väte och syre och inte kan bilda väte-bidningar.
                \item
                $ C $ (Butanal) Butanal har en karbonylgrupp och i och med den en viss laddningsförskjutning. Därför kan butanal bilda dipol-dipol bindingar.
                \item
                $ A $ (Butanol)
                Butanol har en -OH grupp och kan därför bilda vätebindningar med sig själv. Vätebindingar är den starkaste formen.

        \end{enumerate}

        \hfill

    \end{enumerate}

    \item % b)
    linolsyra
    \textit{$ \ch{C17H31COOH}  $}\\
    linolensyra
    \textit{$ \ch{C17H29COOH}  $}\\
    Båda syrorna har samma antal kolatomer men olika antal dubbelbindningar. De är essentiella fettsyror och kallas
    i vardagliga tal för omega-3 och omega-6 syra.

    \begin{enumerate}[label=\textbf{\roman*. }]
        \item % i.
        Vilken information kan man få av iod index över organiska ämnen.

        \textcolor{red}{
            Man kan få reda på hur många dubbelbindningar man har.
        }

        \hfill

        \item % ii.
        \textit{$ \ch{C15H26COOH} $}\\

        % Funktionen:
        f(n(C),n(H)){
            n(C)
        }


        \textit{$ \ch{C15H23COOH} $}\\

    \end{enumerate}

\end{enumerate}

\end{flushleft}
\end{document}
