\documentclass[12pt, letterpaper, twoside]{article}
\usepackage[utf8]{inputenc}
\usepackage{array}
\usepackage{booktabs} % For prettier tables
\usepackage{multirow}
\usepackage{multicol}
\usepackage{ragged2e}
\usepackage{xcolor}

\title{Kemi Vecka 14 lektion 3}
\author{Simon Freiermuth}
\date{15 April 2020}

\begin{document}

\begin{titlepage}
\maketitle
\end{titlepage}
\begin{flushleft}
%$ CH_3-NH_2+HCl\ \rightarrow\ CH_3-NH_3^++Cl^- $
%		  1 2 3 4 5 6 7
\begin{tabular}{ |c|c|c|c|c|c|c| }
\hline
%   1               | 2               | 3            | 4                     | 5                     | 6               | 7
    Molekyl         & $ C_{init} $    & $ n_{init} $ & $ C_{\frac{1}{2}eq} $ & $ n_{\frac{1}{2}eq} $ & $ C_{eq} $      & $ n_{eq} $    \\
\hline
    $ CH_3-NH_2 $   & $ 0.0025 $      & $ 0.05 $     & $ 0.4 $               & $ 0.025 $             & $ - $           & $ - $         \\
\hline
    $ HCl $         & $ - $           & $ - $        & $ - $                 & $ - $                 & $ - $           & $ - $         \\
\hline
    $ CH_3-NH_3^+ $ & $ - $           & $ - $        & $ 0.4 $               & $ 0.025 $             & $ 3.3*10^{-2} $ & $ 0.0025 $    \\
\hline
    $ OH^- $        & $ 6.3*10^{-3} $ & $ 0.126 $    & $ 5.0*10^{-4} $       & $ 8.0*10^{-3} $       & $ 1.0*10^{-8} $ & $ 1.3*10^{-7}$\\
\hline
    Total volym & \multicolumn{2}{c|}{ $ V_{init} = 0.050 $ } & \multicolumn{2}{c|}{ $ V_{\frac{1}{2}eq} = 0.05+0.0125 $ } & \multicolumn{2}{c|}{ $ V_{eq} = 0.050+0.025 $ } \\
\hline
\end{tabular}
\hfill

$ C=\frac{n}{V}\ \rightarrow\ V=\frac{n}{C}\ \rightarrow\ n=C*V $

\hfill \break
$ pH = pK_a + log\Big( \frac{[A^-]}{[HA]} \Big) $

\begin{multicols}{2}
$ K_b = \frac{[CH_3-NH_3^+]*[OH^-]}{[CH_3-NH_2]} $
\newline
\newline
$ log(K_b) = log\Big(\frac{[CH_3-NH_3^+]*[OH^-]}{[CH_3-NH_2]}\Big) $
\newline
\newline
$ -pK_b = log\Big(\frac{[CH_3-NH_3^+]}{[CH_3-NH_2]}\Big)+log([OH^-]) $
\newline
\newline
$ -pK_b = log\Big(\frac{[CH_3-NH_3^+]}{[CH_3-NH_2]}\Big)-pOH $
\newline
\newline
$ pK_b = log\Big(\frac{[CH_3-NH_3^+]}{[CH_3-NH_2]}\Big)+pOH $
\newline
\newline
$ pOH = pK_b+log\Big(\frac{[CH_3-NH_3^+]}{[CH_3-NH_2]}\Big) $
\newline
\columnbreak

$ C_{init}(OH^-) = 10^{-(14-pH)} = 6.3*10^{-3}\ mol/dm^3 $
\newline
\newline
$ C_{\frac{1}{2}eq}(OH^-) = 10^{-(14-pH)} = 5.0*10^{-4}\ mol/dm^3 $
\newline
\newline
$ C_{eq}(OH^-) =\ 10^{-(14-pH)} = 1.0*10^{-8}\ mol/dm^3 $
\end{multicols}
$ pOH_{\frac{1}{2}eq} = pK_b+log(1) $

$ pK_b = 3.4 $
\newline
\newline
Vid ekvivalenspunkten har vi ingen $ CH_3-NH_2 $ kvar, allt har förbrukats av $ HCl $. Vi hällde i $0.025\ dm^3\ 0.1$
molarig lösning $ HCl $
\newline
\newline
$ n_{tillsatt}(HCl)\ =\ n_{eq}(CH_3-NH_3^+) = n_{init}(CH_3-NH_2) $
\newline
\newline
$ n_{tillsatt}(HCl)\ =\ V(HCl)*C(HCl) $

$ n_{tillsatt}(HCl)\ =\ 0.025*0.1=0.0025\ mol $
\newline
\newline
\break
Reaktionen vid Ekvivalenspunkten:
$ CH_3-NH_3^+\ +\ H_2O\ \rightleftharpoons\ CH_3-NH_2\ +\ H_3O^+ $
\newline
\newline
Bromkresol röd är den bästa indikatorn eftersom dess färgomslag är närmast titrerpunkten

\end{flushleft}
\end{document}
