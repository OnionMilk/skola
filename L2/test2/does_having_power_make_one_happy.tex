\documentclass[12pt, letterpaper, twoside]{article}
\usepackage[utf8]{inputenc}
\usepackage[a4paper]{geometry}
\usepackage{array}
\usepackage{booktabs} % For prettier tables
\usepackage{multirow}
\usepackage{multicol}
\usepackage{ragged2e}
\usepackage{xcolor}
\usepackage{gensymb}
\usepackage{fullpage}
\usepackage{hyperref}
\usepackage{amsmath}
\usepackage{scrextend}

\newcommand{\cfootnote}[1]{\footnote{\centering #1}}
\title{Does having power make one happy?}
\author{Simon Freiermuth \\ \href{mailto:simon@freiermuth.org}{simon@freiermuth.org}}
\date{\today}

\begin{document}

%\begin{titlepage}
\maketitle
%\end{titlepage}

\begin{flushleft}

Initially one might assume that more power gives happiness and satifaction to those holding it,
because power gives the possibility to influence or shape things to one's liking.
If one can make decisions then the assumption would be that these decisions lead to greater happiness for oneself.

In \textit{Things fall apart} having power did not make Okonkwo happy, for instance in order to maintain his reputation
he believed that he had to kill Ikemefuna, his adopted son who he liked. For Okonkwo power was the same as 'manliness' and physical power.
These were the strengths that allowed him to build his wealth and his position in the clan.\\

When the missioners built their church in the evil forest and survived the villagers were split in their opinion on how to react.
some took the missionaries survival as a proof that they are right and maybe worth following.
Okonkwo belonged to those against this new influence. The success of the missionaries threatend his position
in particular because his son Nwoye decided to convert to christianity

Having done terrible things to maintain his power first and finally losing it anyways when the english seem to take control of his clan
Okonkwo didn't see any other option than to hang himself

\hfill

McLendon in \textit{Dry September} wasn't happy either, although he seemed to have all the power.
He seems to be frustrated, his glory days as an army officer are clearly over, he is looking for something to do.
In the barbershop he is initiating the lynch mob a bit like a bully on the schoolyard. The group of men is hunting and killing a black man
who is accused of raping a ~40 year old white woman.
When McLendon comes home after the murder he doesn't seem any happier, he is mad at his wife for not cooking food and for not expecting him.

\hfill

In \textit{To Kill a Mockingbird} a lot of factual power lies in the hands of the 'Southern white godess'
when she accuses a black man of having raped her. While she receives a lot of attention it doesn't make her happier,
she remains alone and unloved.



\end{flushleft}

\end{document}
